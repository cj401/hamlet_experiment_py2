\section{Inference}
\label{sec:inference}

We develop a Gibbs sampling algorithm based on the Markov Process with
Failed Jumps representation, augmenting the data with the duration
variables $\bu$, the failed jump attempt count matrix, $\bQ$, as well
as additional auxiliary variables which we will define below.
In this representation the transition matrix is not represented
directly, but is a function of the unscaled transition matrix $\pi$
and the similarity matrix $\bphi$.  The full set of variables is
partitioned into blocks: $\{\gamma, \alpha, \bbeta, \bpi\}$,
$\{\bz, \bu, \bQ, \Lambda\}$, $\{\btheta\}$, and $\{\xi\}$, where $\Lambda$
represents a set of auxiliary variables that will be introduced
below, $\btheta$ represents the emission and state location
parameters (which may be further factored depending on the specific
choice of model), and $\xi$ represents additional
parameters such as any free parameters of the similarity function,
$\Phi$, and any hyperparameters of the emission distribution.


\subsection{Sampling Transition Parameters and Hyperparameters}

The joint posterior over $\gamma$, $\alpha$, $\bbeta$ and $\bpi$
given the other variables will factor as
\begin{equation}
  \label{eq:46}
  p(\gamma, \alpha, \bbeta, \bpi) = p(\gamma) p(\alpha) p(\bbeta \given \gamma) p(\bpi
  \given \alpha, \bbeta)
\end{equation}
where we have omitted the dependence on the augmented data,
$\mathcal{D} = (\bz, \bu, \bQ, \Lambda)$ for conciseness.  
We describe these four factors in reverse order.

\paragraph{Sampling \texorpdfstring{$\bpi$}{pi}} 
Having used data augmentation to simplify the likelihood for $\bpi$ to
the factored conjugate form in \eqref{eq:joint-likelihood}, the individual
$\pi_{jj'}$ are {\it a posteriori} independent Gamma distributed:
\begin{equation}
\pi_{jj'} \given \alpha, \beta_{j'}, \mathcal{D} \stackrel{ind}{\sim}
\Gamm{\alpha\beta_{j'} + n_{jj'} + q_{jj'}}{1 + u_j} \qquad j,j' \geq 1
\end{equation}

\paragraph{Sampling \texorpdfstring{$\bbeta$}{beta}}
\label{sec:sampling-bbeta}
To enable joint sampling of the latent state sequence, we employ a
weak limit approximation to the HDP \cite{johnson2013bayesian}, approximating the stick-breaking
process for $\bbeta$ using a finite Dirichlet distribution with a
finite number of components, $J$, which is larger than we expect to
need.  Due to the product of Gammas form, we can integrate out $\bpi$ 
analytically from $p(\bpi, \mathcal{D} \given \bbeta)$, to obtain the
the marginal likelihood for $\bbeta$.  Together, we have
\begin{align}
\label{eq:beta-prior-likelihood}
  p(\bbeta \given \gamma) = \frac{\Gamma(\gamma /
    J)^J}{\Gamma(\gamma)} \prod_{j} \beta_{j}^{\frac{\gamma}{J} - 1} \qquad
  p(\mathcal{D} \given \bbeta, \alpha) \propto
  \prod_{j=1}^J (1+u_j)^{-\alpha} \prod_{j'}
    \frac{\Gamma(\alpha\beta_{j'} + n_{jj'} +
    q_{jj'})}{\Gamma(\alpha\beta_{j'})} 
\end{align}
 where we have used the fact that the $\beta_j$ sum to 1 to pull out
 terms of the form $(1 + u_j)^{-\alpha\beta_{j'}}$ from the inner
 product in the likelihood.  Following Teh et al. (2006), 
we can introduce auxiliary variables $\{m_{jj'}\}$, with
\begin{equation}
  \label{eq:m-distribution}
  p(m_{jj'} \given \beta_{j'}, \alpha, \mathcal{D}) \stackrel{ind}{\propto}
  s(n_{jj'} + q_{jj'}, m_{jj'}) \alpha^{m_{jj'}}
    \beta_{j'}^{m_{jj'}}
\end{equation}
for integer $m_{jj'}$ ranging between $0$ and $n_{jj'} + q_{jj'}$,
where $s(n,m)$ is an unsigned Stirling number of the first kind.
The normalizing constant in this distribution cancels the ratio of
Gamma functions in the $\bbeta$ likelihood, so, letting $m_{\cdot j} =
\sum_{j'} m_{j'j}$, we obtain simply
\begin{equation}
  \label{eq:beta-posterior}
  \bbeta \given \bM, \gamma \sim \mathrm{Dirichlet}(\frac{\gamma}{J} +
  m_{\cdot 1}, \dots, \frac{\gamma}{J} + m_{\cdot J})
\end{equation}

\paragraph{Sampling Concentration Parameters}
\label{sec:sampling-alpha}
After incorporating $\bM$ into $\mathcal{D}$, we can integrate out
$\bbeta$, from the joint likelihood $p(\bbeta,\mathcal{D} \given \gamma, \alpha)$:
\begin{align}
\label{eq:gamma-marginal-likelihood}
  p(\mathcal{D} \given \alpha, \gamma) \propto
  \alpha^{m_{\cdot\cdot}} e^{-\sum_{j''} \log(1+u_{j''}) \alpha}
  \frac{\Gamma(\gamma)}{\Gamma(\gamma + m_{\cdot\cdot})}
  \prod_j
  \frac{\Gamma(\frac{\gamma}{J} + m_{\cdot
      j})}{\Gamma(\frac{\gamma}{J}) } 
\end{align}
Assume that $\alpha$ and $\gamma$ have Gamma priors.  Then
the update to alpha is conjugate,
\begin{equation}
  \label{eq:alpha-posterior}
  \alpha \given \mathcal{D} \sim \Gamm{a_{\alpha}
    + m_{\cdot\cdot}}{b_\alpha + \sum_j\log(1+u_j)},
\end{equation}
and to simplify the likelihood for $\gamma$, we can introduce a final
set of auxiliary variables, $\br = (r_1, \dots,
r_J)$, $r_j \in \{0,\dots,m_{\cdot j}\}$ and $t \in (0,1)$ with the following distributions:
\begin{align}
  \label{eq:9}
  p(r_j \given m_{\cdot j}, \gamma) \propto s(m_{\cdot j}, r)
    \left(\frac{\gamma}{J}\right)^r \qquad
  p(t \given m_{\cdot\cdot} \gamma) \propto 
    t^{\gamma - 1} (1-t)^{m_{\cdot\cdot} - 1}.
\end{align}
The normalizing constants are ratios of Gamma functions, which cancel
those in \eqref{eq:gamma-marginal-likelihood}, so that
\begin{equation}
  \label{eq:18}
  \gamma \given \mathcal{D} \sim \Gamm{a_{\gamma} + r_{\cdot}}{b_{\gamma} - \log(t)}
\end{equation}

\subsection{Sampling \texorpdfstring{$\bz$}{the latent state sequece} and the auxiliary variables}
\label{sec:sampling-z_t}

We sample the hidden state sequence, $\bz$, jointly with the auxiliary
variables, which consist of $\bu$, $\bQ$, $\bM$, $\br$ and $t$.  The
joint conditional distribution of these variables is defined directly
by the generative model:
\begin{align}
  \label{eq:19}
  p(\mathcal{D}) = p(\bz) p(\bu \given \bz) p(\bQ \given
  \bu) p(\bM \given \bz, \bQ) p(\br \given \bM) p(t
  \given\bM)
\end{align}
Since we are conditioning on the transition matrix, we can
sample the entire sequence $\bz$ at once with the forward-backward algorithm,
as in an ordinary HMM, or its corresponding generalization if we are
using the HSMM variant.  Having done this, we can sample $\bu$, $\bQ$, $\bM$,
$\br$ and $t$ from their forward distributions.

\subsection{Sampling state and emission parameters}
\label{sec:sampling-eta}

Depending on the application, the similarities $\{\phi_{jj'}\}$ may be based
directly on the emission distributions, or may be based on a separate set
of variables.  In the experiments described below we assume the
former.  For simplicity, we denote the collection of these variables
by $\btheta$.  We have two likelihood components:
\begin{align}
  \label{eq:65}
  p(\bz, \bQ \given \btheta) \propto \prod_{j}\prod_{j'}
  \phi_{jj'}^{n_{jj'}}(1-\phi_{jj'})^{q_{jj'}} \qquad
  p(\bY \given \bz, \btheta) = \prod_{t=1}^T f(\by_t; \theta_{z_t})
\end{align}
where proportionality is with respect to variation in $\btheta$.

% \section{Inference}
% \label{sec:inference}

% I develop a Gibbs sampling algorithm based on the Markov Process with
% Failed Jumps representation, augmenting the data with the duration
% variables $\bu$, the failed jump attempt count matrix, $\bQ$, as well
% as additional auxiliary variables which we will define below.
% In this representation the transition matrix is not modeled
% directly, but is a function of the unscaled transition matrix $\pi$
% and the similarity matrix $\bphi$.  The full set of variables is
% partitioned into three blocks: $\{\gamma, \alpha, \bbeta, \bpi\}$,
% $\{\bz, \bu, \bQ, \Lambda\}$, and $\{\btheta\}$, where $\Lambda$
% represents a set of auxiliary variables that will be introduced
% below.  The variables in each block are sampled jointly 
% conditioned on the other two blocks.

% Since we are representing the transition matrix of the Markov chain
% explicitly, we approximate the stick-breaking process that produces
% $\bbeta$ using a finite Dirichlet distribution with a number of 
% components larger than we expect to need, forcing the remaining 
% components to have zero weight.  
% Let $J$ indicate the maximum number of states.  Then,
% we approximate \eqref{eq:beta} with
% \begin{equation}
%   \label{eq:28}
%   \bbeta \given \gamma \sim \mathrm{Dirichlet}(\gamma / J, \dots,
%   \gamma / J)
% \end{equation}
% This distribution converges weakly to the Stick-Breaking Process as $J \to
% \infty$.  In practice, $J$ is large enough when the vast majority of the
% probability mass in $\bbeta$ is allocated to a strict subset of
% components, or when the latent state sequence $\bz$ never uses all $J$
% available states, indicating that the data is well described by a number of
% states less than $J$.

% \subsection{Sampling Transition Parameters and Hyperparameters}
% \label{sec:sampling-pi}

% The joint conditional over $\gamma$, $\alpha$, $\bbeta$ and $\bpi$
% given $\bz$, $\bu$, $\bQ$, $\Lambda$ and $\btheta$ will factor as
% \begin{equation}
%   \label{eq:46}
%   p(\gamma, \alpha, \bbeta, \bpi \given \bz, \bu, \bQ, \Lambda, \btheta) = p(\gamma \given
%   \Lambda) p(\alpha \given \Lambda) p(\bbeta \given \gamma, \Lambda) p(\bpi
%   \given \alpha, \bbeta, \btheta, \bz)
% \end{equation}
% I will derive these four factors in reverse order.

% \subsubsection{Sampling \texorpdfstring{$\bpi$}{pi}}

% The entries in $\bpi$ are conditionally independent given $\alpha$ and
% $\bbeta$, so we have the prior
% \begin{equation}
%   \label{eq:47}
%   p(\bpi \given \bbeta, \alpha) = \prod_{j} \prod_{j'}
%   \Gamma(\alpha\beta_{j'})^{-1}
%   \pi_{jj'}^{\alpha\beta_{j'} - 1} \exp(-\pi_{jj'}),
% \end{equation}
% and the likelihood given augmented data $\{\bz, \bu, \bQ\}$ given by
% \eqref{eq:joint-likelihood}.  Combining these, we have
% \begin{equation}
%   \label{eq:61}
%   p(\bpi, \bz, \bu, \bQ \given \bbeta, \alpha, \btheta) =
%   \prod_{j} u_j^{n_{j\cdot} + q_{j\cdot}
%   - 1}\prod_{j'} 
%   \Gamma(\alpha\beta_{j'})^{-1} \pi_{jj'}^{\alpha\beta_{j'} + n_{jj'}
%     + q_{jj'} - 1} e^{-(1 + u_j)
%     \pi_{jj'}} \phi_{jj'}^{n_{jj'}} (1-\phi_{jj'})^{q_{jj'}} (q_{jj'}!)^{-1}
% \end{equation}
% Conditioning on everything except $\bpi$, we get
% \begin{align}
%   \label{eq:24}
%   p(\bpi \given \bQ, \bu, \bZ, \bbeta, \alpha, \btheta) &\propto \prod_j
%   \prod_{j'} \pi_{jj'}^{\alpha\beta_{j'} + n_{jj'} + q_{jj'} - 1}
%   \exp(-(1 + u_j)\pi_{jj'})
% \end{align}
% and thus we see that the $\pi_{jj'}$ are conditionally independent
% given $\bu$, $\bZ$ and $\bQ$, and distributed according to
% \begin{align}
%   \label{eq:25}
%   \pi_{jj'} \given n_{jj'}, q_{jj'}, \beta_{j'}, \alpha \stackrel{ind}{\sim}
%   \Gamm{\alpha\beta_{j'} + n_{jj'} + q_{jj'}}{1 + u_j}
% \end{align}


% \subsubsection{Sampling \texorpdfstring{$\bbeta$}{beta}}
% \label{sec:sampling-bbeta}

% Consider the conditional distribution of $\bbeta$ having
% integrated out $\bpi$.  The prior density of $\bbeta$ from
% \eqref{eq:28} is
% \begin{equation}
%   \label{eq:62}
%   p(\bbeta \given \gamma) =
%   \frac{\Gamma(\gamma)}{\Gamma(\frac{\gamma}{J})^J} \prod_{j}
%   \beta_j^{\frac{\gamma}{J} - 1}
% \end{equation}
% After integrating out $\bpi$ in \eqref{eq:61}, we have
% \begin{align}
%   p(\bz, \bu, \bQ \given \bbeta, \alpha, \gamma, \btheta) &=
%   \prod_{j=1}^J u_{j} ^{-1}
%   \prod_{j'=1}^J u^{n_{jj'} + q_{jj'} - 1}(1 +
%   u_j)^{-(\alpha\beta_{j'} + n_{jj'} + q_{jj'})}
%   \\
%   &\qquad \qquad \times \frac{\Gamma(\alpha\beta_{j'} + n_{jj'} +
%     q_{jj'})}{\Gamma(\alpha\beta_{j'})} \phi_{jj'}^{n_{jj'}}(1-\phi_{jj'})^{q_{jj'}}
%   (q_{jj'}!)^{-1} \\
%   &= \prod_{j=1}^J \Gamma(n_{j\cdot})^{-1} u_j^{-1}(1+u_j)^{-\alpha}
%   \left(\frac{u_j}{1+u_j}\right)^{n_{j\cdot} + q_{j\cdot}} \\ &\qquad
%   \qquad \times \prod_{j' =
%     1}^J \frac{\Gamma(\alpha\beta_{j'} + n_{jj'} +
%     q_{jj'})}{\Gamma(\alpha\beta_{j'})} \phi_{jj'}^{n_{jj'}}(1-\phi_{jj'})^{q_{jj'}}
%   (q_{jj'}!)^{-1}
% \end{align}
%  where we have used the fact that the $\beta_j$ sum to 1.  Therefore
% \begin{align}
%   p(\bbeta \given \bz, \bu, \bQ, \alpha, \gamma, \btheta) &\propto \prod_{j=1}^J
%   \beta_j^{\frac{\gamma}{J} - 1} \prod_{j'=1}^J \frac{\Gamma(\alpha\beta_{j'} +
%     n_{jj'} + q_{jj'})}{\Gamma(\alpha\beta_{j'})}.
% \end{align}

% Following \cite{teh2006hierarchical}, we can write the ratios of Gamma functions
% as polynomials in $\beta_j$, as
% \begin{equation}
%   \label{eq:31}
%   p(\bbeta \given \bz, \bu, \bQ, \alpha, \gamma, \btheta) \propto \prod_{j=1}^J
%   \beta_j^{\frac{\gamma}{J} - 1} \prod_{j'=1}^{J} \sum_{m_{jj'} = 1}^{n_{jj'}}
%   s(n_{jj'} + q_{jj'}, m_{jj'}) (\alpha \beta_{j'})^{m_{jj'}}
% \end{equation}
% where $s(m,n)$ is an unsigned Stirling number of the first kind.
% This admits an augmented data representation, where we introduce a
% random matrix $\bM = (m_{jj'})_{1 \leq j,j' \leq J}$, whose
% entries are conditionally independent given $\bbeta$, $\bQ$ and $\bz$, with
% \begin{equation}
%   \label{eq:32}
%   p(m_{jj'} = m \given \beta_{j'}, \alpha, n_{jj'}, q_{jj'}) =
%   \frac{s(n_{jj'} + q_{jj'}, m) \alpha^{m}
%     \beta_{j'}^{m}}{\sum_{m'=0}^{n_{jj'} + q_{jj'}} s(n_{jj'} +
%   q_{jj'}, m') \alpha^{m'} \beta_{j'}^{m'}}
% \end{equation}
% for integer $m$ ranging between $0$ and $n_{jj'} + q_{jj'}$.  Note
% that $s(n,0) = 0$ if $n > 0$, $s(0,0) = 1$ and $s(0,m) = 0$ if $m > 0$.
% Then, we have joint distribution
% \begin{equation}
%   \label{eq:33}
%   p(\bbeta, \bM \given \bz, \bu, \bQ, \alpha, \gamma, \btheta) \propto \prod_{j=1}^J
%   \beta_j^{\frac{\gamma}{J} - 1} \prod_{j'=1}^{J} s(n_{jj'} + q_{jj'}, m_{jj'}) \alpha^{m_{jj'}} \beta_{j'}^{m_{jj'}}
% \end{equation}
% which yields \eqref{eq:31} when marginalized over $\bM$.  Again discarding
% constants in $\bbeta$ and regrouping yields
% \begin{equation}
%   \label{eq:34}
%   p(\bbeta \given \bM, \bZ, \bu, \btheta, \alpha, \gamma) \propto \prod_{j=1}^J
%   \beta_j^{\frac{\gamma}{J} + m_{\cdot j}- 1}
% \end{equation}
% which is Dirichlet:
% \begin{equation}
%   \label{eq:38}
%   \bbeta \given \bM, \gamma \sim \mathrm{Dirichlet}(\frac{\gamma}{J} +
%   m_{\cdot 1}, \dots, \frac{\gamma}{J} + m_{\cdot J})
% \end{equation}

% \subsubsection{Sampling Concentration Parameters}
% \label{sec:sampling-alpha}
% Assume that $\alpha$ and $\gamma$ have Gamma priors, with
% \begin{align}
%   \label{eq:42}
%   p(\alpha) &= \frac{b_{\alpha}^{a_{\alpha}}}{\Gamma(a_{\alpha})}
%   \alpha^{a_{\alpha} - 1} \exp(-b_{\alpha}\alpha) \\
%   p(\gamma) &= \frac{b_{\gamma}^{a_\gamma}}{\Gamma(a_{\gamma})}
%   \gamma^{a_{\gamma - 1}} \exp(-b_{\gamma}\gamma)
% \end{align}

% Having integrated out $\bpi$, we have
% \begin{align}
%   p(\bbeta, \bz, \bu, \bQ, \bM \given \alpha, \gamma, \btheta) &=
%   \frac{\Gamma(\gamma)}{\Gamma(\frac{\gamma}{J})^J} \alpha^{m_{\cdot\cdot}} \prod_{j=1}^J \beta_j^{\frac{\gamma}{J} +
%     m_{\cdot j} - 1}\Gamma(n_{j\cdot})^{-1} u_j^{-1}(1+u_j)^{-\alpha}
%   \left(\frac{u_j}{1+u_j}\right)^{n_{j\cdot} + q_{j\cdot}} \\ &\qquad
%   \qquad \times \prod_{j' =
%     1}^J s(n_{jj'} + q_{jj'}, m_{jj'}) \phi_{jj'}^{n_{jj'}}(1-\phi_{jj'})^{q_{jj'}}
%   (q_{jj'}!)^{-1}
% \end{align}
% We can also integrate out $\bbeta$, to yield
% \begin{align}
%   p(\bz, \bu, \bQ, \bM \given \alpha, \gamma, \btheta) &=
%   \alpha^{m_{\cdot\cdot}} e^{-\sum_{j''} \log(1+u_{j''}) \alpha}
%   \frac{\Gamma(\gamma)}{\Gamma(\gamma + m_{\cdot\cdot})} \\ &\qquad
%   \qquad \times \prod_j
%   \frac{\Gamma(\frac{\gamma}{J} + m_{\cdot
%       j})}{\Gamma(\frac{\gamma}{J}) \Gamma(n_{j\cdot})} u_j^{-1}
%   \left(\frac{u_j}{1+u_j}\right)^{n_{j\cdot} + q_{j\cdot}} \\ &\qquad
%   \qquad \times \prod_{j' =
%     1}^J s(n_{jj'} + q_{jj'}, m_{jj'}) \phi_{jj'}^{n_{jj'}}(1-\phi_{jj'})^{q_{jj'}}
%   (q_{jj'}!)^{-1}
% \end{align}
% demonstrating that $\alpha$ and $\gamma$ are independent given $\btheta$
% and the augmented data, with
% \begin{equation}
%   \label{eq:43}
%   p(\alpha \given \bz, \bu, \bQ, \bM, \btheta) \propto
%   \alpha^{a_{\alpha} + m_{\cdot\cdot}}\exp(-(b_\alpha + \sum_{j}\log(1+u_j))\alpha)
% \end{equation}
% and
% \begin{align}
%   \label{eq:8}
%   p(\gamma \given \bz, \bu, \bQ, \bM, \btheta) &\propto \gamma^{a_{\gamma - 1}}
%   \exp(-b_{\gamma}\gamma) \frac{\Gamma(\gamma)\prod_{j=1}^J
%     \Gamma(\frac{\gamma}{J} + m_{\cdot j})}{\Gamma(\frac{\gamma}{J})^J\Gamma(\gamma + m_{\cdot\cdot})}
% \end{align}
% So we see that
% \begin{equation}
%   \label{eq:44}
%   \alpha \given \bz, \bu, \bQ, \bM, \btheta \sim \Gamm{a_{\alpha}
%     + m_{\cdot\cdot}}{b_\alpha + \sum_j\log(1+u_j)}
% \end{equation}
% To sample $\gamma$, we introduce a new set of auxiliary variables, $\br = (r_1, \dots,
% r_j)$ and $t$ with the following distributions:
% \begin{align}
%   \label{eq:9}
%   p(r_j = r \given m_{\cdot j}, \gamma) &=
%   \frac{\Gamma(\frac{\gamma}{J})}{\Gamma(\frac{\gamma}{J}
%     + m_{\cdot j})} s(m_{\cdot j}, r)
%     \left(\frac{\gamma}{J}\right)^r \qquad r  = 1, \dots, m_{\cdot j} \\
%   p(t \given m_{\cdot\cdot} \gamma) &= \frac{\Gamma(\gamma +
%     m_{\cdot\cdot})}{\Gamma(\gamma) \Gamma(m_{\cdot\cdot})} t^{\gamma
%     - 1} (1-t)^{m_{\cdot\cdot} - 1} \qquad t \in (0,1)
% \end{align}
% so that
% \begin{align}
%   \label{eq:10}
%   p(\gamma, \br, t \given \bM) &\propto \gamma^{a_{\gamma - 1}}
%   \exp(-b_{\gamma}\gamma) t^{\gamma - 1}(1-t)^{m_{\cdot\cdot} +
%     q_{\cdot} - 1} \prod_{j=1}^J s(m_{\cdot j} + q_j, r_j)
%   \left(\frac{\gamma}{J}\right)^{r_j}
% \end{align}
% and
% \begin{align}
%   \label{eq:11}
%   p(\gamma \given \br, t) \propto \gamma^{a_\gamma +
%     r_{\cdot} - 1} \exp(-(b_{\gamma} - \log(t)) \gamma),
% \end{align}
% which is to say
% \begin{equation}
%   \label{eq:18}
%   \gamma \given \br, t, \bz, \bu, \bQ, \bM, \btheta \sim \Gamm{a_{\gamma} + r_{\cdot}}{b_{\gamma} - \log(t)}
% \end{equation}

% \subsubsection{Summary}

% I have made the following additional assumptions about the generative
% model in this section:
% \begin{equation}
%   \label{eq:100}
%   \gamma \sim \Gamm{a_{\gamma}}{b_{\gamma}} \qquad \alpha \sim \Gamm{a_{\alpha}}{b_{\alpha}}
% \end{equation}

% The joint conditional over $\gamma$, $\alpha$, $\bbeta$ and $\bpi$
% given $\bz$, $\bu$, $\bQ$, $\bM$, $\br$, $t$ and $\btheta$ factors as
% \begin{equation}
%   \label{eq:46}
%   p(\gamma, \alpha, \bbeta, \bpi \given \bz, \bu, \bQ, \br, t,
%   \btheta) = p(\gamma \given \br, t) p(\alpha \given \bu, \bM) p(\bbeta
%   \given \gamma, \bM) p(\bpi \given \alpha, \bbeta, \bz, \bu, \bQ)
% \end{equation}
% where
% \begin{align}
%   \label{eq:64}
%   \gamma \given \br, t &\sim \Gamm{a_{\gamma} + r_{\cdot}}{b_{\gamma} -
%     \log(t)} \\
%   \alpha \given \bu, \bM &\sim \Gamm{a_{\alpha} +
%     m_{\cdot\cdot}}{b_{\alpha} + \sum_j \log(1 + u_j)} \\
%   \bbeta \given \gamma, \bM &\sim \mathrm{Dirichlet}(\frac{\gamma}{J} + m_{\cdot 1},
%   \dots, \frac{\gamma}{J} + m_{\cdot J}) \\
%   \pi_{jj'} \given \alpha, \beta_{j'}, \bz, \bu, \bQ
%   &\stackrel{ind}{\sim} \Gamm{\alpha\beta_{j'} + n_{jj'} + q_{jj'}}{1 +
%   u_j}
% \end{align}


% \subsection{Sampling \texorpdfstring{$\bz$}{the latent state sequece} and the auxiliary variables}
% \label{sec:sampling-z_t}

% The hidden state sequence, $\bz$, is sampled jointly with the auxiliary
% variables, which consist of $\bu$, $\bM$, $\bQ$, $\br$ and $t$.  The
% joint conditional distribution of these variables is defined directly
% by the generative model:
% \begin{align}
%   \label{eq:19}
%   p(\bz, \bu, \bQ, \bM, \br, t \given \bpi, \bbeta, \alpha, \gamma,
%   \btheta) &= p(\bz \given \bpi, \btheta) p(\bu \given \bz, \bpi, \btheta) p(\bQ \given
%   \bu, \bpi, \btheta) p(\bM \given
%   \bz, \bQ, \alpha, \bbeta) \\
%   &\qquad \times p(\br \given
%   \gamma, \bM) p(t \given \gamma, \bM)
% \end{align}
% Since we are representing the transition matrix explicitly, we can
% sample the entire sequence $\bz$ at once with the forward-backward algorithm,
% as in an ordinary HMM (or, if we are employing the HSMM variant
% described in Sec. \ref{sec:an-hsmm-modification}, then we can use the
% modified message passing scheme for HSMMs described by
% \cite{johnson2013bayesian}.  
% Having done this, we can sample $\bu$, $\bQ$, $\bM$,
% $\br$ and $t$ from their forward distributions.  To summarize,
% we have
% \begin{align}
%   \label{eq:48}
%   u_j \given \bZ, \bpi, \btheta &\stackrel{ind}{\sim}
%   \Gamm{n_{j\cdot}}{\sum_{j'} \pi_{jj'}\phi_{jj'}} \\
%   q_{jj'} \given u_j, \pi_{jj'}, \phi_{jj'} &\stackrel{ind}{\sim}
%   \Pois{u_j(1 - \phi_{jj'})\pi_{jj'}} \\
%   m_{jj'} \given n_{jj'}, q_{jj'}, \beta_{j'}, \alpha &\stackrel{ind}{\sim}
%   \frac{\Gamma(\alpha\beta_j)}{\Gamma(\alpha\beta_j + n_{jj'} +
%     q_{jj'})}\sum_{m=1}^{n_{jj'} + q_{jj'}} s(n_{jj'} + q_{jj'}, m) \alpha^m \beta_{j'}^m \delta_{m}
%   \\
%   r_j \given m_{\cdot j}, \gamma &\stackrel{ind}{\sim}
%   \frac{\Gamma(\frac{\gamma}{J})}{\Gamma(\frac{\gamma}{J} + m_{\cdot
%       j})} \sum_{r=1}^{m_{j\cdot}} s(m_{\cdot j}, r)
%   \left(\frac{\gamma}{J}\right)^r \delta_r \\
%   t \given \gamma, \bM &\sim \Beta{\gamma}{m_{\cdot\cdot}}
% \end{align}

% \subsection{Sampling state and emission parameters}
% \label{sec:sampling-eta}

% The state parameters, $\btheta$, influence the transition matrix,
% $\bpi$ and the auxiliary vector $q$ through the similarity matrix
% matrix $\bphi$, and also control the emission distributions.
% We have likelihood factors
% \begin{align}
%   \label{eq:65}
%   p(\bz, \bQ \given \btheta) &\propto \prod_{j}\prod_{j'}
%   \phi_{jj'}^{n_{jj'}}(1-\phi_{jj'})^{q_{jj'}} \\
%   p(\bY \given \bz, \btheta) &= \prod_{t=1}^T f(\by_t; \theta_{z_t})
% \end{align}
% where proportionality is with respect to variation in $\btheta$.

% The parameter space for the hidden states, 
% the associated prior $H$ on $\btheta$, and the similarity function
% $\Phi$, is application-specific, but we consider here the case where a state,
% $\theta_j$, consists of a finite-length binary vector, motivated by the
% application of inferring the set of relevant entities in each sentence of
% a text document.

% Let $\theta_j = (\theta_{j1}, \dots, \theta_{jD})$, with $\theta_{jd} = 1$
% indicating presence of feature $d$ in context state $j$, and
% $\theta_{jd} = 0$ indicating absence.  Of course, in this case,
% the set of possible states is finite, and so on its face it may
% seem that a nonparametric model is unnecessary.  However, if $D$ is
% reasonably large, it is likely that most of the $2^D$ possible states
% are vanishingly unlikely (and, in fact, the number of observations may
% well be less than $2^D$), and so we would like a model that encourages
% the selection of a sparse set of states.  Moreover, there may be more
% than one state with the same $\theta$, but with different transition dynamics.

% \subsubsection{Sampling \texorpdfstring{$\btheta$}{latent state vectors}}
% \label{sec:sampling-eta}

% In principle, $H$ can be any distribution over binary vectors, but we
% will suppose for simplicity that it can be factored into $D$
% independent coordinate-wise Bernoulli variates.  Let $\mu_d$ be the
% Bernoulli parameter for the $d$th coordinate.

% We require a similarity function, $\Phi(\theta_j, \theta_{j'})$, which 
% varies between 0 to 1, and is equal to 1 if and only if $\theta_j =
% \theta_{j'}$.  A natural choice in this setting is the Laplacian kernel:
% \begin{align}
%   \label{eq:39}
%   \phi_{jj'} &= \Phi(\theta_j, \theta_{j'}) = \exp(-\lambda \Delta_{jj'})
% \end{align}
% where $\Delta_{jj'd} = \abs{\theta_{jd} - \theta_{j'd}}$, $\Delta_{jj'} =
% \sum_{d=1}^D \Delta_{jj'}$ is the Hamming
% distance between $\theta_j$ and $\theta_{j'}$,
% and $\lambda \geq 0$ (if $\lambda = 0$, the $\phi_{jj'}$
% are identically 1, and so do not have any influence, reducing the
% model to an ordinary HDP-HMM).

% Let
% \begin{align}
%   \label{eq:68}
%   \phi_{jj'-d} &= \exp(-\lambda(\Delta_{jj'} - \Delta_{jj'd}))
% \end{align}
% so that $\phi_{jj'} = \phi_{jj'-d} e^{-\lambda\Delta_{jj'd}}$.

% Since the matrix $\bphi$ is assumed to be symmetric, we have
% \begin{align}
%   \label{eq:70}
%   \frac{p(\bz, \bQ \given \theta_{jd}  = 1, \btheta\setminus\theta_{jd}
%     )}{p(\bz, \bQ \given \theta_{jd}  = 0, \btheta\setminus\theta_{jd} )}
%   &\propto \prod_{j' \neq j}
%   \frac{e^{-\lambda(n_{jj'} + n_{j'j})\abs{1 - \theta_{j'd}}}(1 -
%     \phi_{jj'-d} e^{-\lambda\abs{1 - \theta_{j'd}}})^{q_{jj'} +
%       q_{j'j}}}{e^{-\lambda(n_{jj'} + n_{j'j})\abs{\theta_{j'd}}}
%     (1-\phi_{jj'-d}e^{-\lambda\abs{\theta_{j'd}}})^{q_{jj'} +
%       q_{j'j}}} \\
%   &= \label{eq:71} e^{-\lambda(c_{jd0} - c_{jd1})}
%   \prod_{j' \neq j} \left(\frac{1 - \phi_{jj'-d}e^{-\lambda}}{1-\phi_{jj'-d}}\right)^{(-1)^{\theta_{j'd}}(q_{jj'} + q_{j'j})}
% \end{align}
% where $c_{jd0}$ and $c_{jd1}$ are the number of successful jumps to or
% from state $j$, to or from states with a 0 or 1, respectively, in
% position $d$.  That is,
% \begin{equation}
%   \label{eq:72}
%   c_{jd0} = \sum_{\{j' \given \theta_{j'd} = 0\}} n_{jj'} + n_{j'j}\qquad c_{jd1} = \sum_{\{j' \given \theta_{j'd} = 1\}} n_{jj'} + n_{j'j}
% \end{equation}

% Therefore, we can Gibbs sample $\theta_{jd}$ from its conditional
% posterior Bernoulli distribution given the rest of $\btheta$, where
% we compute the Bernoulli parameter via the log-odds
% \begin{align}
%   \label{eq:77}
%   &\log\left(\frac{p(\theta_{jd} = 1 \given \bY, \bz, \bQ, \btheta \setminus
%     \theta_{jd})}{p(\theta_{jd} = 0 \given \bY, \bz, \bQ, \btheta
%     \setminus \theta_{jd})}\right) = \log\left(\frac{p(\theta_{jd} =
%   1) p(\bz, \bQ \given \theta_{jd} = 1, \btheta \setminus
%   \theta_{jd}) p(\bY \given \bz, \theta_{jd} = 1, \btheta \setminus \theta_{jd})}{p(\theta_{jd} = 0) p(\bz, \bQ \given \theta_{jd} = 0,
%   \btheta \setminus \theta_{jd}) p(\bY \given \bz, \theta_{jd} = 0,
%   \btheta \setminus \theta_{jd})}\right) \\ & \qquad = \log\left(\frac{\mu_d}{1 - \mu_d}\right)
%   + (c_{jd1} - c_{jd0}) \lambda +
%     \sum_{j' \neq j}
%   (-1)^{\theta_{j'd}}(q_{jj'} + q_{j'j})\log\left(\frac{1 -
%       \phi_{jj'}^{(-d)}e^{-\lambda}}{1-\phi_{jj'}^{(-d)}}\right) \\ &
%   \qquad \qquad + \sum_{\{t \given z_t = j\}} \log\left(\frac{f(\by_t;
%       \theta_{jd} = 1, \theta_j \setminus \theta_{jd})}{f(\by_t;
%       \theta_{jd} = 0, \theta_j \setminus \theta_{jd})}\right)
% \end{align}

% Suppose also that the observed data $\bY$ consists of a $T \times K$
% matrix, where the $t$th row $\by_t = (y_{t1}, \dots,
% y_{tK})^{\mathsf{T}}$ is a $K$-dimensional feature vector associated
% with time $t$, and let $\bW$ be a $D \times K$ weight matrix
% with $k$th column $\bw_k$, such that 
% \begin{equation}
%   \label{eq:74}
%   f(\by_t; \theta_j) = g(\by_t; \bW^{\mathsf{T}} \theta_j)
% \end{equation}
% for a suitable parametric function $g$.  I will assume for simplicity
% that $g$ factors as
% \begin{equation}
%   \label{eq:73}
%   g(\by_t; \bW^{\mathsf{T}} \theta_j) = \prod_{k=1}^K g_k(y_{tk}; \bw_k \cdot \theta_j)
% \end{equation}
% Define $x_{tk} = \bw_k \cdot \theta_{z_{t}}$, and
% $x_{tk}^{(-d)} = \bw_k^{-d} \cdot \theta_{z_{t}}^{-d}$, where
% $\theta_{j}^{-d}$ and $\bw_{k}^{-d}$ are $\theta_{j}$ and $\bw_k$, respectively, with
% the $d$th coordinate removed.  Then
% \begin{equation}
%   \label{eq:76}
%   \log\left(\frac{f(\by_t; \theta_{jd} = 1, \theta_j \setminus
%     \theta_{jd})}{f(\by_t; \theta_{jd} = 0, \theta_j \setminus \theta_{jd})}\right) =
%   \sum_{k=1}^K \log\left(\frac {g_k(y_{tk};
%     x_{tk}^{(-d)} + w_{dk})}{g_k(y_{tk};
%     x_{tk}^{(-d)})}\right)
% \end{equation}
% If $g_k(y; x)$ is a Normal density with mean $x$ and unit variance, then
% \begin{equation}
%   \label{eq:91}
%   \log\left(\frac {g_k(y_{tk};
%     x_{tk}^{(-d)} + w_{dk})}{g_k(y_{tk};
%     x_{tk}^{(-d)})}\right) = -w_{dk}(y_{tk} - x_{tk}^{(-d)} + \frac{1}{2}w_{dk})
% \end{equation}

% \subsubsection{Sampling \texorpdfstring{$\bmu$}{mu}}
% \label{sec:sampling-bmu}

% Sampling the $\mu_d$ is straightforward with a Beta prior.  Suppose
% \begin{equation}
%   \label{eq:92}
%   \mu_d \stackrel{ind}{\sim} \Beta{a_\mu}{b_\mu}
% \end{equation}
% Then, conditioned on $\btheta$ the $\mu_d$ are independent with
% \begin{equation}
%   \label{eq:93}
%   \mu_d \given \btheta \sim \Beta{a_\mu + \sum_{j} \theta_{jd}}{b_\mu +
%   \sum_{j} (1 - \theta_{jd})}
% \end{equation}

% \subsubsection{Sampling \texorpdfstring{$\lambda$}{kernel decay rate}}
% \label{sec:sampling-lambda}

% The parameter $\lambda$ governs the connection between $\btheta$ and
% $\bphi$.  Writing \eqref{eq:65} in terms of $\lambda$ and the difference matrix
% $\boldsymbol{\Delta} = (\Delta_{jj'})_{1 \leq j,j' \leq J}$ gives
% \begin{equation}
%   \label{eq:88}
%   p(\bz, \bQ \given \lambda, \btheta) \propto \prod_{j}\prod_{j'}
%   e^{-\lambda \Delta_{jj'} n_{jj'}}(1-e^{-\lambda\Delta_{jj'}})^{q_{jj'}} 
% \end{equation}
% Put an $\Exp{b_{\lambda}}$ prior on $\lambda$, so that
% \begin{equation}
%   \label{eq:88}
%   p(\lambda \given \bz, \bQ, \btheta) \propto
%   e^{-(b_{\lambda} + \sum_{j}\sum_{j'} \Delta_{jj'} n_{jj'})\lambda} \prod_{j}\prod_{j'}
%   (1-e^{-\lambda\Delta_{jj'}})^{q_{jj'}}
% \end{equation}
% This density is log-concave, with
% \begin{equation}
%   \label{eq:90}
%   -\frac{d^2\log(p(\lambda \given \bz, \bQ,
%     \btheta))}{d\lambda^2} = \sum_{\{(j,j') \given
%     \Delta_{jj'} > 0\}}
%   \frac{\Delta_{jj'}^2 q_{jj'}
%     e^{\lambda\Delta_{jj'}}}{(e^{\lambda\Delta_{jj'}} - 1)^2} > 0
% \end{equation}
% and so we can use Adaptive Rejection Sampling \cite{gilks1992adaptive}
% to sample from it.  The relevant $h$ and $h'$, representing the log
% density and its first derivative, respectively, are
% \begin{align}
%   \label{eq:94}
%   h(\lambda) &= 
%   -(b_{\lambda} + \sum_{\{(j,j') \given \Delta_{jj'} > 0\}} \Delta_{jj'} n_{jj'})\lambda +
%   \sum_{\{(j,j') \given \Delta_{jj'} > 0\}} q_{jj'} \log(1 - e^{-\lambda\Delta_{jj'}}) \\
%   h'(\lambda) &= -(b_{\lambda} + \sum_{\{(j,j') \given \Delta_{jj'} > 0\}} \Delta_{jj'}
%   n_{jj'}) + \sum_{\{(j,j') \given \Delta_{jj'} > 0\}}
%   \frac{q_{jj'}\Delta_{jj'}}{e^{\lambda\Delta_{jj'}} - 1}
% \end{align}

% \subsubsection{Sampling \texorpdfstring{$\bW$}{weights}}

% Conditioned on the state matrix $\btheta$ and the data matrix $\bY$, the weight matrix $\bW$ can be sampled as well using standard methods
% for Bayesian regression problems.  For example, suppose that the
% weights are {\em a priori} i.i.d. Normal:
% \begin{equation}
%   \label{eq:78}
%   p(\bW) = \prod_{k=1}^K\prod_{d=1}^D \mathcal{N}(w_{dk} \given 0,\sigma^2_0)
% \end{equation}
% and the likelihood is
% \begin{equation}
%   \label{eq:79}
%   g_k(y; x) = \mathcal{N}(y \given x,1)
% \end{equation}
% Then it is a standard result from Bayesian linear modeling that
% \begin{equation}
%   \label{eq:80}
%   p(\bW \given \btheta, \bY) = \prod_{k=1}^K
%   \mathcal{N}\left(\left(\sigma_0^2 \mathbf{I} + \btheta^{\mathsf{T}} \btheta
%     \right)^{-1}\btheta^{\mathsf{T}}\by_k, \sigma_0^2 \mathbf{I} + \btheta^{\mathsf{T}} \btheta\right)
% \end{equation}

% If one or more output features, say $\by_k$, is binary, we can adopt a
% probit model where we introduce a latent data vector $\by^*_k$ for
% each such $k$, and assume
% \begin{equation}
%   \label{eq:83}
%   p(\by^*_{k} \given \bx_k) = \prod_{t} \mathcal{N}(y^*_{tk} \given x_{tk}, 1)
% \end{equation}
% and 
% \begin{equation}
%   \label{eq:84}
%   y_{tk} = \begin{cases}
%     0,& y^*_{tk} \leq 0 \\
%     1,& y^*_{tk} > 0
%   \end{cases}
% \end{equation}
% And so, after marginalizing over $\by^*_k$
% \begin{equation}
%   \label{eq:81}
%   p(\by_k \given \bx_k) = \prod_{t=1}^T F(x_{tk})^{y_{tk}} (1 - F(x_{tk}))^{1 - y_{tk}}
% \end{equation}
% where $F$ is the standard Normal CDF, since
% \begin{equation}
%   \label{eq:85}
%   \int_{0}^{\infty} dy^*_{tk} \mathcal{N}(y^*_{tk} \given x_{tk}, 1) =
%   \int_{-x_{tk}}^{\infty} dy^*_{tk} \mathcal{N}(y^*_{tk} \given 0, 1)
%   = 1 - F(-x_{tk}) = F(x_{tk})
% \end{equation}
% Then, conditioned on $x_{tk}$ and $y_{tk}$, we can sample $y^*_{tk}$ 
% from a Normal distribution left- or right-truncated at 0:
% \begin{equation}
%   \label{eq:82}
%   p(y^*_{tk} \given x_{tk}, y_{tk}) = \begin{cases}
%     \mathcal{N}(x_{tk}, 1) I(y^{*}_{tk} \leq 0), & y_{tk} = 0 \\
%     \mathcal{N}(x_{tk}, 1) I(y^*_{tk} > 0), & y_{tk} = 1
%   \end{cases}
% \end{equation}
% Conditioned on the $y^*_{tk}$ and $\btheta$, the weights are
% distributed as in \eqref{eq:80}.

% \subsubsection{Summary}
% \label{sec:summary}

% I have made the following assumptions about the representation of the
% hidden states and observed data in this subsection:
% (1) $\btheta$ consists of $D$ binary features (2) the similarity function $\Phi$
% is the Laplacian kernel with respect to Hamming distance with
% decay parameter $\lambda$, and (3) $\bY$ consists of $K$ continuous or 
% binary features associated with each time step $t$.  In addition, we
% make the following distributional assumptions:
% \begin{align}
% \mu_d &\stackrel{i.i.d}{\sim} \Beta{a_{\mu}}{b_{\mu}} \\
% \lambda &\sim \Exp{b_{\lambda}} \\
% \theta_{jd} \given \bmu &\stackrel{ind}{\sim} \Bern{\mu_d} \\
% \bW \sim \Norm{0}{\sigma^2_0 \mathbf{I}}
% y^*_{tk} \given \bW, \bz, \btheta &\stackrel{ind}{\sim} \Norm{x_{tk}}{1} \\
% y_{tk} &= \begin{cases}
%   y^*_{tk}, & \text{if $k$ is a continuous feature} \\
%   \mathbb{I}(y^*_{tk} > 0) & \text{if $k$ is a binary feature}
% \end{cases}
% \end{align}
% where we have defined
% \begin{align}
%   \label{eq:102}
%   x_{tk} = \bw_k \cdot \theta_{z_t}
% \end{align}

% I introduce Gibbs blocks corresponding to (1) each $\theta_{jd}$
% individually, (2) the vector $\bmu$, (3) the decay parameter
% $\lambda$, (4) the weight matrix $\bW$, and (5) the latent data
% $\bY^*$ associated with binary features.  We have
% \begin{align}
%   \label{eq:101}
%   \theta_{jd} \given \btheta \setminus \theta_{jd}, \bz, \bQ, \bmu,
%   \lambda, \bW, \bY^* &\sim
%   \Bern{\frac{e^{\zeta_{jd}}}{1 + e^{\zeta_{jd}}}} \\
%   \mu_d \given \btheta, \dots &\stackrel{ind}{\sim} \Beta{a_\mu + \sum_{j} \theta_{jd}}{b_\mu +
%   \sum_{j} (1 - \theta_{jd})} \\
% p(\lambda \given \bz, \bQ, \btheta, \dots) &\propto e^{-(b_{\lambda} + \sum_{j}\sum_{j'} \Delta_{jj'} n_{jj'})\lambda} \prod_{j}\prod_{j'}
%   (1-e^{-\lambda\Delta_{jj'}})^{q_{jj'}} \\
%   \bw_k \given \btheta, \bY^{*}, \dots &\stackrel{ind}{\sim}
%   \Norm{(\sigma_0^2 \mathbf{I} + \btheta^{\mathsf{T}}
%     \btheta)^{-1}\btheta^{\mathsf{T}}\by^*_k}{\sigma_0^2 \mathbf{I} +
%     \btheta^{\mathsf{T}} \btheta} \\
%   \by^*_{tk} \given \bX, \bY, \dots &\stackrel{ind}{\sim} \begin{cases}
%     \Norm{x_{tk}}{1} \mathbb{I}(y^*_{tk} \leq 0), & y_{tk} = 0 \\
%     \Norm{x_{tk}}{1} \mathbb{I}(y^*_{tk} > 0), & y_{tk}= 1
%   \end{cases}
% \end{align}
% where $\Delta_{jj'} = \norm{\theta_j - \theta_j'}_{L_1}$ and
% \begin{align}
% \zeta_{jd} &= \log\left(\frac{\mu_d}{1 - \mu_d}\right)
%   + (c_{jd1} - c_{jd0}) \lambda +
%     \sum_{j' \neq j}
%   (-1)^{\theta_{j'd}}(q_{jj'} + q_{j'j})\log\left(\frac{1 -
%       \phi_{jj'}^{(-d)}e^{-\lambda}}{1-\phi_{jj'}^{(-d)}}\right)
%   \notag \\ & \qquad - \sum_{\{t \given z_t = j\}} \sum_{k=1}^K
%   w_{dk}(y^*_{tk} - x_{tk}^{(-d)} + \frac{1}{2}w_{dk})
% \end{align}
% All distributions can be sampled from directly except for $\lambda$, which
% requires Adaptive Rejection Sampling, with the equations
% \begin{align}
%   h(\lambda) &= 
%   -(b_{\lambda} + \sum_{\{(j,j') \given \Delta_{jj'} > 0\}} \Delta_{jj'} n_{jj'})\lambda +
%   \sum_{\{(j,j') \given \Delta_{jj'} > 0\}} q_{jj'} \log(1 - e^{-\lambda\Delta_{jj'}}) \\
%   h'(\lambda) &= -(b_{\lambda} + \sum_{\{(j,j') \given \Delta_{jj'} > 0\}} \Delta_{jj'}
%   n_{jj'}) + \sum_{\{(j,j') \given \Delta_{jj'} > 0\}}
%   \frac{q_{jj'}\Delta_{jj'}}{e^{\lambda\Delta_{jj'}} - 1}
% \end{align}
