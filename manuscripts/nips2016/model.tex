\section{An HDP-HMM With Local Transitions}
\label{sec:model}
We wish to add to the transition model the concept of a transition to
a ``nearby'' state, where nearness of $j$ and $j'$ is a function of
$\theta_j$ and $\theta_{j'}$.  In order to accomplish this, we first
consider an alternative construction of the transition distributions,
based on the Normalized Gamma Process.

\subsection{A Normalized Gamma Process representation of the HDP-HMM}
\label{sec:normalized-gamma}

We can define a random measure, $\mu = \sum_{j=1}^{\infty} \pi_j \delta_{\theta_j}$, where 
\begin{align}
  \pi_j \stackrel{ind}{\sim} \Gamm{w_j}{1} \qquad T =
  \sum_{j=1}^{\infty} \pi_j \qquad
  \tilde{\pi}_j = \frac{\pi_j}{T} \label{eq:20}
\end{align}
and subject to the constraint that $\sum_{j\geq 1} w_j < \infty$.  It
follows \cite{paisley2012discrete, favaro2013mcmc} that $\mu$ is distributed as a Dirichlet
Process with base measure $\bw = \sum_{j=1}^{\infty} w_j \delta_{\theta_j}$.
If we draw $\bbeta$ from a stick-breaking process and then draw a
series $\{\mu_m\}_{m=1}^M$ of
i.i.d. random measures from the above process, setting $\bw =
\alpha\bbeta$ for some $\alpha > 0$, then
this defines a Hierarchical Dirichlet Process.  If, moreover, there is
one $\mu$ associated with every state $j$, then we obtain the
transition prior for the HDP-HMM, where
\begin{align}
  \label{eq:50}
  p(z_t \given z_{t-1}, \bpi) = \tilde{\pi}_{z_{t-1}z_t}
\end{align}

\subsection{Promoting ``Local" Transitions}
\label{sec:prom-local-trans}

In the preceding formulation, the $\theta_j$ and the $\pi_{jj'}$ are independent
conditioned on the top-level measure.  Our goal is to relax this
assumption, in order to incorporate possible prior knowledge
that certain ``location'' pairs, $(\theta_j, \theta_{j'})$, are more likely than others to
produce large transition weights (i.e., states adjacent in time should
tend to be similar).  This can be accomplished by scaling the elements
$\pi_{jj'}$ by a function of $(\theta_{j}, \theta_{j'})$ prior to
normalization, or equivalently letting the Gamma distribution have
a proximity-dependent rate parameter.  

Let $\Phi: \Omega \times \Omega \to [0,\infty)$ represent a
``similarity function'', and define a collection of random variables
$\{\phi_{jj'}\}_{j,j' \geq 1}$ according to $\phi_{jj'} = \phi(\theta_j, \theta_j')$.
We can then generalize \eqref{eq:20} to
\begin{align}
  \pi_{jj'} \given \bbeta, \btheta \sim \Gamm{\alpha \beta_{j'}}{\phi_{jj'}^{-1}} \qquad
  T_j = \sum_{j'=1}^{\infty} \pi_{jj'} \qquad \tilde{\pi}_{jj'} = \frac{\pi_{jj'}}{T_j}
\end{align}
so that the prior mean of $\pi_{jj'}$ is
$\alpha\beta_{j'}\phi_{jj'}$.  Since a similarity between one object
and another should not exceed the similarity between an object and
itself, and since a constant rescaling of the similarity will be
absorbed in normalization, we will assume that $0 \leq \phi_{jj'} \leq 1$ for all $j$
and $j'$.

\subsection{The HDP-HMM-LT as the Marginalization of
a Markov Jump Process with ``Failed'' Jumps}
\label{sec:dist-based-filt}

We can gain stronger intuition, as well as simplify posterior
inference, by representing the HDP-HMM-LT described in the last section
as a continuous time Markov jump process where holding times have been
integrated out.  In particular, suppose that some of the attempts to jump
from one state to another fail, and the failure probability
increases as a function of the ``distance'' between the states.

Let $\Phi$ be defined as in the last section, and let 
$\bbeta$, $\btheta$ and $\bpi$ be defined as in the Normalized Gamma
Process representation of the ordinary HDP-HMM (so,
  $\pi_{jj'} \given \bbeta, \btheta \sim \Gamm{\alpha \beta_{j'}}{1}$).
Now suppose that when the process is in state $j$, jumps to state
$j'$ are made at rate $\pi_{jj'}$.  This defines a continuous-time
Markov Process where the off-diagonal elements of the transition rate
matrix are the off diagonal elements of $\bpi$.  In addition,
self-jumps are allowed, and occur with rate $\pi_{jj}$.   If we only
observe the jumps and not the durations between jumps, this is an
ordinary Markov chain.  If we do not observe the jumps themselves, but
instead an observation is generated once per jump from a distribution that depends
on the state being jumped to, then we have an ordinary HMM.

We modify this process as follows.  
Suppose that each jump attempt from state $j$ to state $j'$ has a
chance of failing, which is an increasing function of the ``distance''
between the states.  In particular, let the success probability be
$\phi_{jj'}$ (recall that we assumed above that $0 \leq \phi_{jj'}
\leq 1$ for all $j,j'$).  Then, the rate of successful jumps from $j$
to $j'$ is $\pi_{jj'}\phi_{jj'}$, and the corresponding rate of unsuccessful jump
attempts is $\pi_{jj'}(1-\phi_{jj'})$.  We denote the overall rate of successful jumps
while in state $j$ by $T_j := \sum_{j'} \pi_{jj'} \phi_{jj'}$.
Given that the process is in state $j$ at discretized time $t$ (that is,
$z_{t} = j$), the probability that the first successful jump is to
state $j'$ (that is, $z_{t+1} = j'$) 
is proportional to the rate of successful jump attempts to $j'$, which
is $\pi_{jj'}\phi_{jj'}$.  The holding time, $\tau_{t-1}$, is
independent of $z_{t+1}$ and is distributed $\mathsf{Exp}(T_j)$.  The
total time spent in state $j$ given that it is visited $n_{j}$ times,
is then
\begin{equation}
u_j \given \bz, \bpi, \btheta \stackrel{ind}{\sim} \Gamm{n_{j\cdot}}{T_j}
\end{equation}
During this period there will be $q_{jj'}$
unsuccessful attempts to jump to state $j'$, where $q_{jj'}$ is distributed $\mathsf{Pois}(u_j
\pi_{jj'}(1 - \phi_{jj'}))$.  Incorporating
$\bu = \{u_j\}$ as augmented data simplifies the likelihood for the transition
parameters, yielding
\begin{align}
  L(\bpi, \bphi \given \bz, \bu, \bQ) &= \left(\prod_{t=1}^T p(z_{t} \given
    z_{t-1}, \bpi, \bphi)\right) \prod_{j} p(u_j \given \bz, \bpi, \bphi)
  \prod_{j'} p(q_{jj'} \given u_j \pi_{jj'}, \phi_{jj'}) \notag \\
  &\propto \prod_{j} \prod_{j'} \pi_{jj'}^{n_{jj'} + q_{jj'}} \phi_{jj'}^{n_{jj'}}
  (1-\phi_{jj'})^{q_{jj'}} e^{-\pi_{jj'}u_j} \label{eq:joint-likelihood}
\end{align}

\subsection{An HDP-HSMM-LT modification}
\label{sec:an-hsmm-modification}

We note that it is trivial to modify the HDP-HMM-LT to allow for
non-Geometric duration distributions, by simply fixing the diagonal
elements of $\bpi$ to be zero, allowing $D_t$ observations to be
emitted $i.i.d.$ according to $F(\theta_{z_t})$ at jump $t$, where $D_t$ is drawn
from a state-specific duration distribution, and sampling the
latent state sequence using a message passing algorithm suited for
HSMMs \cite{johnson2013bayesian}.  Inference for the $\bphi$ matrix
is not affected, since the diagonal elements are assumed to be 1.
Unlike in the original representation of the HDP-HSMM, there is no need to introduce
additional auxiliary variables as a result of this modification, due
to the presence of the (continuous) durations, $\bu$, which were
already needed to account for the normalization of the $\bpi$.

% \subsection{Summary}
% \label{sec:model-summary}

% We have defined the following augmented generative model for the
% HDP-H(S)MM-LT:
% \begin{align}
%   \label{eq:96}
%   \bbeta &\sim \mathrm{GEM}(\gamma) \\
%   \theta_j &\stackrel{i.i.d}{\sim} H \\
%   \pi_{jj'} \given \bbeta, \btheta &\sim \Gamm{\alpha \beta_{j'}}{1}
%   \\
%   z_{t} \given z_{t-1}, \bpi, \btheta &\sim \sum_{j}
%   \left(\frac{\pi_{z_{t-1}j}\phi_{z_{t-1}j}}{\sum_{j'}
%     \pi_{z_{t-1}j'}\phi_{z_{t-1}j'}}\right)\delta_j \\
%   u_j \given \bz, \bpi, \btheta &\stackrel{ind}{\sim}
%   \Gamm{n_{j\cdot}}{\sum_{j'} \pi_{jj'}\phi_{jj'}} \\
%   q_{jj'} \given \bu, \bpi, \btheta &\stackrel{ind}{\sim}
%   \Pois{u_j(1 - \phi_{jj'})\pi_{jj'}} \\
%   \label{eq:likelihood} \by_t \given \bz, \btheta &\sim F(\theta_{z_t})
% \end{align}

% If we are using the HSMM variant, then we simply fix $\pi_{jj}$ to 0
% for each $j$, draw
% \begin{align}
%   \label{eq:97}
%   \omega_j &\stackrel{i.i.d}{\sim} G \\
%   D_t \given \bz &\stackrel{ind}{\sim} g(\omega_{z_t}),
% \end{align}
% set
% \begin{equation}
%   \label{eq:98}
%   z^*_s = z_{\max\{T \given s \leq \sum_{t=1}^T D_t\}}
% \end{equation}
% and replace \eqref{eq:likelihood} with
% \begin{equation}
%   \label{eq:likelihood-hsmm} \by_s \given \bz, \btheta \sim F(\theta_{z^*_s})
% \end{equation}


% \section{An HDP-HMM With Local Transitions}

% The goal is to add to the transition model the concept of a transition to
% a ``nearby'' state, where nearness of $j$ and $j'$ is possibly a function of
% $\theta_j$ and $\theta_{j'}$.  In order to accomplish this, we first
% consider an alternative construction of the transition distributions,
% based on the Normalized Gamma Process representation of the Dirichlet
% Process \cite{ferguson1973bayesian}.

% \subsection{A Normalized Gamma Process representation of the HDP-HMM}
% \label{sec:normalized-gamma}

% Define a random measure, $\mu = \sum_{j=1}^{\infty} \pi_j \delta_{\theta_j}$, where 
% \begin{align}
%   \pi_j &\stackrel{ind}{\sim} \Gamm{w_j}{1} \label{eq:17}\\
%   T &= \sum_{j=1}^{\infty} \pi_j \label{eq:18}\\
%   \tilde{\pi}_j &= \frac{\pi_j}{T}   \label{eq:16}\\
%   \theta_j &\stackrel{i.i.d}{\sim} H \label{eq:19}
% \end{align}
% and subject to the constraint that $\sum_{j\geq 1} w_j < \infty$,
% which ensures that $T < \infty$ almost surely.  As
% shown by Paisley et al. (2011), for fixed $\{w_j\}$ and $\{\theta_j\}$, $\mu$ is distributed as a Dirichlet
% Process with base measure $\bw = \sum_{j=1}^{\infty} w_j \delta_{\theta_j}$.
% If we draw $\bbeta$ from a stick-breaking process and then draw a
% series $\{\mu_m\}_{m=1}^M$ of
% i.i.d. random measures from the above process, setting $\bw =
% \alpha\bbeta$ for some $\alpha > 0$, then
% this defines a Hierarchical Dirichlet Process.  If, moreover, there is
% one $\mu_m$ associated with every state $j$, then we obtain the
% HDP-HMM.

% We can thus write
% \begin{align}
%   \bbeta &\sim \mathsf{GEM}(\gamma)   \label{eq:20} \\
%   \theta_j &\stackrel{i.i.d.}{\sim} H \label{eq:21}\\
%   \pi_{jj'} &\stackrel{ind}{\sim} \Gamm{\alpha \beta_{j'}}{1} \label{eq:22}\\
%   T_j &= \sum_{j'=1}^{\infty} \pi_{jj'} \\
%   \tilde{\pi}_{jj'} &= \frac{\pi_{jj'}}{T_j} \label{eq:23},
% \end{align}
% where $\gamma$ and $\alpha$ are prior concentration hyperparameters
% for the two DP levels, where
% \begin{align}
%   \label{eq:50}
%   p(z_t \given z_{t-1}, \bpi) = \tilde{\pi}_{z_{t-1}z_t}
% \end{align}
% and the observed data
% $\{y_t\}_{t\geq 1}$ distributed as
% \begin{equation}
%   \label{eq:24}
%   y_t \given z_t \stackrel{ind}{\sim} F(\theta_{z_t})
% \end{equation}
% for some family, $F$ of probability measures indexed by values of $\theta$.

% \subsection{Promoting ``Local" Transitions}
% \label{sec:prom-local-trans}

% In the preceding formulation, the $\theta_j$ and the $\pi_{jj'}$ are independent
% conditioned on the top-level measure.  Our goal is to relax this
% assumption, in order to allow for prior knowledge
% that certain ``locations'', $\theta_j$, are more likely than others to
% produce large weights.  This can be accomplished by letting the rate
% parameter in the distribution of the $\pi_{jj'}$
% be a function of $\theta_j$ and $\theta_{j'}$.  
% Let $\Phi: \Omega \times \Omega \to [0,\infty)$ represent a
% ``similarity function'', and define a collection of random variables
% $\{\phi_{jj'}\}_{j,j' \geq 1}$ according to
% \begin{equation}
%   \phi_{jj'} = \phi(\theta_j, \theta_j')
% \end{equation}
% We can then generalize \eqref{eq:20}-\eqref{eq:23} to
% \begin{align}
%   \bbeta &\sim \mathrm{GEM}(\gamma) \\
%   \theta_j &\stackrel{i.i.d}{\sim} H \\
%   \pi_{jj'} \given \bbeta, \btheta &\sim \Gamm{\alpha \beta_{j'}}{\phi_{jj'}^{-1}} \\
%   T_j &= \sum_{j'=1}^{\infty} \pi_{jj'} \\
%   \tilde{\pi}_{jj'} &= \frac{\pi_{jj'}}{T_j}
% \end{align}
% so that the expected value of $\pi_{jj'}$ is
% $\alpha\beta_{j'}\phi_{jj'}$.  Since a similarity between one object
% and another should not exceed the similarity between an object and
% itself, we will assume that $\phi_{jj'} \leq B < \infty$ for all $j$
% and $j'$, with equality holding iff $j = j'$.  Moreover, there 
% is no loss of generality by taking $B = 1$, since a constant rescaling of
% $\phi_{jj'}$ gets absorbed in the normalization.

% The above model is equivalent to simply drawing the $\pi_{jj'}$ as in
% \eqref{eq:20} and scaling each one by $\phi_{jj'}$ prior to
% normalization.

% Unfortunately, this formulation complicates inference significantly,
% as the introduction of non-constant rate parameters to the prior on
% $\bpi$ destroys the conjugacy between $\bpi$ and $\bz$, and worse, the
% conditional likelihood function for $\bpi$ contains an infinite
% sum of the elements in a row, rendering all entries within a row
% mutually dependent.

% \subsection{The HDP-HMM-LT as a continuous-time 
% Markov Jump Process with ``failed'' jumps}
% \label{sec:dist-based-filt}

% We can gain stronger intuition, as well as simplify posterior
% inference, by re-casting the HDP-HMM-LT described in the last section
% as a continuous time Markov Jump Process where some of the attempts to jump
% from one state to another fail, and where the failure probability
% increases as a function of the ``distance'' between the states.

% Let $\Phi$ be defined as in the last section, and let 
% $\bbeta$, $\btheta$ and $\bpi$ be defined as in the Normalized Gamma
% Process representation of the ordinary HDP-HMM.  That is,
% \begin{align}
%   \label{eq:beta} \bbeta &\sim \mathrm{GEM}(\gamma) \\
%   \theta_j &\stackrel{i.i.d}{\sim} H \\
%   \pi_{jj'} \given \bbeta, \btheta &\sim \Gamm{\alpha \beta_{j'}}{1}
% \end{align}
% Now suppose that when the process is in state $j$, jumps to state
% $j'$ are made at rate $\pi_{jj'}$.  This defines a continuous-time
% Markov Process where the off-diagonal elements of the transition rate
% matrix are the off diagonal elements of $\bpi$.  In addition,
% self-jumps are allowed, and occur with rate $\pi_{jj}$.   If we only
% observe the jumps and not the durations between jumps, this is an
% ordinary Markov chain, whose transition matrix is obtained by
% appropriately normalizing $\bpi$.  If we do not observe the jumps themselves, but
% instead an observation is generated once per jump from a distribution that depends
% on the state being jumped to, then we have an ordinary HMM.

% I modify this process as follows.  
% Suppose that each jump attempt from state $j$ to state $j'$ has a
% chance of failing, which is an increasing function of the ``distance''
% between the states.  In particular, let the success probability be
% $\phi_{jj'}$ (recall that we assumed above that $0 \leq \phi_{jj'}
% \leq 1$ for all $j,j'$).  Then, the rate of successful jumps from $j$
% to $j'$ is $\pi_{jj'}\phi_{jj'}$, and the corresponding rate of unsuccessful jump
% attempts is $\pi_{jj'}(1-\phi_{jj'})$.  To see this, denote by
% $N_{jj'}$ the total number of jump attempts to $j'$ in a unit
% interval of time spent in state $j$.  Since we are assuming the
% process is Markovian, the total number of attempts is $\Pois{\pi_{jj'}}$
% distributed.  Conditioned on $N_{jj'}$, $n_{jj'}$ will be successful, where
% \begin{equation}
%   \label{eq:51}
%   n_{jj'} \given N_{jj'} \sim \Binom{N_{jj'}}{\phi_{jj'}}
% \end{equation}
% It is easy to show (and well known) that the marginal distribution of
% $n_{jj'}$ is $\Pois{\pi_{jj'}\phi_{jj'}}$, and the marginal
% distribution of $\tilde{q}_{jj'} := N_{jj'} - n_{jj'}$ is
% $\Pois{\pi_{jj'}(1-\phi_{jj'})}$.  The rate of successful jumps
% from state $j$ overall is then $T_j := \sum_{j'} \pi_{jj'} \phi_{jj'}$.

% Let $t$ index jumps, so that $z_t$ indicates the $t$th state visited
% by the process (couting self-jumps as a new time step).  Given
% that the process is in state $j$ at discretized time $t-1$ (that is,
% $z_{t-1} = j$), it is a standard property of Markov Processes that 
% the probability that the first successful jump is to state $j'$ (that is, $z_{t} = j'$) 
% is proportional to the rate of successful attempts to 
% $j'$, which is $\pi_{jj'}\phi_{jj'}$.  

% Let $\tau_{t}$ indicate the time elapsed between the $t$th and 
% and $t-1$th successful jump (where we assume that the first
% observation occurs when the first successful jump from a distinguished initial
% state is made).  We have
% \begin{equation}
%   \label{eq:52}
%   \tau_t \given z_{t-1} \sim \Exp{T_{z_{t-1}}}
% \end{equation}
% where $\tau_t$ is independent of $z_{t}$.

% During this period, there will be $\tilde{q}_{j't}$ unsuccessful attempts to
% jump to state $j'$, where
% \begin{equation}
%   \label{eq:53}
%   \tilde{q}_{j't} \given z_{t-1} \sim \Pois{\tau_t \pi_{z_{t-1}j'}(1-\phi_{z_{t-1}j'})}
% \end{equation}

% Define the following additional variables
% \begin{align}
%   \label{eq:56}
%     \mathcal{T}_j &= \{t \given z_{t-1} = j\} \\
%     q_{jj'} &= \sum_{t \in \mathcal{T}_j}
%     \tilde{q}_{j't} \\
%     u_j &= \sum_{t \in \mathcal{T}_j} \tau_t 
% \end{align}
% and let $\bQ = (q_{jj'})_{j,j' \geq 1}$ be the matrix of unsuccessful
% jump attempt counts, and $\bu = (u_j)_{j \geq 1}$ be the vector of
% the total times spent in each state.

% Since each of the $\tau_t$ with $t \in \mathcal{T}_j$ are
% i.i.d. $\Exp{T_j}$, we get the marginal distribution
% \begin{equation}
% u_j \given \bz, \bpi \btheta \stackrel{ind}{\sim} \Gamm{n_{j\cdot}}{T_j}
% \end{equation}
% by the standard property that sums of i.i.d. Exponential distributions
% has a Gamma distribution with shape equal to the number of variates in
% the sum, and rate equal to the rate of the individual exponentials.  
% Moreover, since the $\tilde{q}_{j't}$ with $t \in \mathcal{T}_j$ 
% are Poisson distributed, the total number of failed
% attempts in the total duration $u_j$ is
% \begin{equation}
%   \label{eq:60}
%   q_{jj'} \stackrel{ind}{\sim} \Pois{u_j\pi_{jj'}(1-\phi_{jj'})}.
% \end{equation}

% Thus if we marginalize out the individual $\tau_t$ and
% $\tilde{q}_{j't}$, we have a joint distribution
% over $\bz$, $\bu$, and $\bQ$, conditioned on the transition rate
% matrix $\bpi$ and the success probability matrix $\bphi$, which is
% \begin{align}
%   \label{eq:54}
%   p(\bz, \bu, \bQ \given \bpi, \btheta) &= \left(\prod_{t=1}^T p(z_{t} \given
%     z_{t-1})\right) \prod_{j} p(u_j \given \bz, \bpi, \btheta)
%   \prod_{j'} p(q_{jj'} \given u_j \pi_{jj'}, \phi_{jj'}) \\
%   &= \left(\prod_{t} \frac{\pi_{z_{t-1}z_t}\phi_{z_{t-1}z_t}}{T_{z_{t-1}}}\right) \prod_{j}
%   \frac{T_j^{n_{j\cdot}}}{\Gamma(n_{j\cdot})} u_j^{n_{j\cdot} - 1}
%   e^{-T_j u_j} \\ &\qquad\qquad\times
%   \prod_{j'} e^{-u_j\pi_{jj'}(1-\phi_{jj'})} u_j^{q_{jj'}}
%   \pi_{jj'}^{q_{jj'}} (1-\phi_{jj'})^{q_{jj'}} (q_{jj'}!)^{-1} \\
%   &= \prod_{j} \Gamma(n_{j\cdot})^{-1} u_j^{n_{j\cdot} + q_{j\cdot}-1}
%   \\ &\qquad\qquad \times \prod_{j'}
%   \pi_{jj'}^{n_{jj'} + q_{jj'}} \phi_{jj'}^{n_{jj'}}
%   (1-\phi_{jj'})^{q_{jj'}} e^{-\pi_{jj'}\phi_{jj'}u_j}
%   e^{-\pi_{jj'}(1-\phi_{jj'})u_j} (q_{jj'}!)^{-1} \\
%   &\label{eq:joint-likelihood} = \prod_{j} \Gamma(n_{j\cdot})^{-1} u_j^{n_{j\cdot} + q_{j\cdot}-1} \prod_{j'}
%   \pi_{jj'}^{n_{jj'} + q_{jj'}} \phi_{jj'}^{n_{jj'}}
%   (1-\phi_{jj'})^{q_{jj'}} e^{-\pi_{jj'}u_j} (q_{jj'}!)^{-1}
% \end{align}

% \subsection{An HDP-HSMM-LT modification}
% \label{sec:an-hsmm-modification}

% Note that it is trivial to modify the HDP-HMM-LT to allow the
% number of observations generated each time a state is visited to have
% a distribution which is not Geometric, by simply fixing the diagonal
% elements of $\bpi$ to be zero, and allowing $D_t$ observations to be
% emitted $i.i.d.$ $F(\theta_{z_t})$ at jump $t$, where
% \begin{equation}
%   \label{eq:95}
%   D_t \given \bz \stackrel{ind}{\sim} g(\omega_{z_t}) \qquad \omega_j
%   \stackrel{i.i.d}{\sim} G
% \end{equation}
% The likelihood then includes the additional term for the $D_t$, and
% the only inference step which is affected is that instead of sampling
% $\bz$ alone, we sample $\bz$ and the $D_t$ jointly, by defining
% \begin{equation}
%   z^*_s = z_{\max\{T \given s \leq \sum_{t=1}^T D_t\}}
% \end{equation}
% where $s$ ranges over the number of observations, 
% and associating a $\by_s$ with each $z^*_s$.
% Inferences about $\bphi$ are not affected, since the diagonal
% elements are assumed to be 1 anyway.

% This is the same construction used in the Hierarchical Dirichlet
% Process Hidden Semi-Markov Model (HDP-HSMM;
% \cite{johnson2013bayesian}).  
% Unlike in the standard representation of the HDP-HSMM,
% however, there is no need to introduce
% additional auxiliary variables as a result of this modification, due
% to the presence of the (continuous) durations, $\bu$, which were
% already needed to account for the normalization of the $\bpi$.

% \subsection{Summary}
% \label{sec:model-summary}

% I have defined the following augmented generative model for the
% HDP-H(S)MM-LT:
% \begin{align}
%   \label{eq:96}
%   \bbeta &\sim \mathrm{GEM}(\gamma) \\
%   \theta_j &\stackrel{i.i.d}{\sim} H \\
%   \pi_{jj'} \given \bbeta, \btheta &\sim \Gamm{\alpha \beta_{j'}}{1}
%   \\
%   z_{t} \given z_{t-1}, \bpi, \btheta &\sim \sum_{j}
%   \left(\frac{\pi_{z_{t-1}j}\phi_{z_{t-1}j}}{\sum_{j'}
%     \pi_{z_{t-1}j'}\phi_{z_{t-1}j'}}\right)\delta_j \\
%   u_j \given \bz, \bpi, \btheta &\stackrel{ind}{\sim}
%   \Gamm{n_{j\cdot}}{\sum_{j'} \pi_{jj'}\phi_{jj'}} \\
%   q_{jj'} \given \bu, \bpi, \btheta &\stackrel{ind}{\sim}
%   \Pois{u_j(1 - \phi_{jj'})\pi_{jj'}} \\
%   \label{eq:likelihood} \by_t \given \bz, \btheta &\sim F(\theta_{z_t})
% \end{align}

% If we are using the HSMM variant, then we simply fix $\pi_{jj}$ to 0
% for each $j$, draw
% \begin{align}
%   \label{eq:97}
%   \omega_j &\stackrel{i.i.d}{\sim} G \\
%   D_t \given \bz &\stackrel{ind}{\sim} g(\omega_{z_t}),
% \end{align}
% for chosen $G$ and $g$, set
% \begin{equation}
%   \label{eq:98}
%   z^*_s = z_{\max\{T \given s \leq \sum_{t=1}^T D_t\}}
% \end{equation}
% and replace \eqref{eq:likelihood} with
% \begin{equation}
%   \label{eq:likelihood-hsmm} \by_s \given \bz, \btheta \sim F(\theta_{z^*_s})
% \end{equation}
